\documentclass[]{article}
\usepackage{lmodern}
\usepackage{amssymb,amsmath}
\usepackage{ifxetex,ifluatex}
\usepackage{fixltx2e} % provides \textsubscript
\ifnum 0\ifxetex 1\fi\ifluatex 1\fi=0 % if pdftex
  \usepackage[T1]{fontenc}
  \usepackage[utf8]{inputenc}
\else % if luatex or xelatex
  \ifxetex
    \usepackage{mathspec}
  \else
    \usepackage{fontspec}
  \fi
  \defaultfontfeatures{Ligatures=TeX,Scale=MatchLowercase}
\fi
% use upquote if available, for straight quotes in verbatim environments
\IfFileExists{upquote.sty}{\usepackage{upquote}}{}
% use microtype if available
\IfFileExists{microtype.sty}{%
\usepackage{microtype}
\UseMicrotypeSet[protrusion]{basicmath} % disable protrusion for tt fonts
}{}
\usepackage[margin=1in]{geometry}
\usepackage{hyperref}
\hypersetup{unicode=true,
            pdftitle={The Statistical Analysis Report on The Effect of Auditory distractions on Cognitive Flexibility.},
            pdfborder={0 0 0},
            breaklinks=true}
\urlstyle{same}  % don't use monospace font for urls
\usepackage{graphicx,grffile}
\makeatletter
\def\maxwidth{\ifdim\Gin@nat@width>\linewidth\linewidth\else\Gin@nat@width\fi}
\def\maxheight{\ifdim\Gin@nat@height>\textheight\textheight\else\Gin@nat@height\fi}
\makeatother
% Scale images if necessary, so that they will not overflow the page
% margins by default, and it is still possible to overwrite the defaults
% using explicit options in \includegraphics[width, height, ...]{}
\setkeys{Gin}{width=\maxwidth,height=\maxheight,keepaspectratio}
\IfFileExists{parskip.sty}{%
\usepackage{parskip}
}{% else
\setlength{\parindent}{0pt}
\setlength{\parskip}{6pt plus 2pt minus 1pt}
}
\setlength{\emergencystretch}{3em}  % prevent overfull lines
\providecommand{\tightlist}{%
  \setlength{\itemsep}{0pt}\setlength{\parskip}{0pt}}
\setcounter{secnumdepth}{0}
% Redefines (sub)paragraphs to behave more like sections
\ifx\paragraph\undefined\else
\let\oldparagraph\paragraph
\renewcommand{\paragraph}[1]{\oldparagraph{#1}\mbox{}}
\fi
\ifx\subparagraph\undefined\else
\let\oldsubparagraph\subparagraph
\renewcommand{\subparagraph}[1]{\oldsubparagraph{#1}\mbox{}}
\fi

%%% Use protect on footnotes to avoid problems with footnotes in titles
\let\rmarkdownfootnote\footnote%
\def\footnote{\protect\rmarkdownfootnote}

%%% Change title format to be more compact
\usepackage{titling}

% Create subtitle command for use in maketitle
\providecommand{\subtitle}[1]{
  \posttitle{
    \begin{center}\large#1\end{center}
    }
}

\setlength{\droptitle}{-2em}

  \title{The Statistical Analysis Report on The Effect of Auditory distractions
on Cognitive Flexibility.}
    \pretitle{\vspace{\droptitle}\centering\huge}
  \posttitle{\par}
    \author{}
    \preauthor{}\postauthor{}
    \date{}
    \predate{}\postdate{}
  

\begin{document}
\maketitle

\hypertarget{abstract}{%
\section{Abstract:}\label{abstract}}

From this report, we found out that general backgroup noises would
deficit cognitive flexibity,since the earphones with noise cancelling
resulted in a higher cognitive flexibity than the earphones that did not
cancel noises.Also,subjects have higher cognitive flexibity in a quiet
enviroment than listenning to music with lyrics. However,there is an
exception to classical music, it has higher cognitive flexibility than
quiet condition.

From our final model, the average time difference between the on and off
Stroop test for a subject under quiet background noise with in-ear and
noise cancelling earphones was around 6.47 seconds. In general, the
earphones with noise cancelling resulted in lower time differences than
the earphones that did not cancel noises. Moreover, we found that the
average time differences were smaller for order two and order three.
This phenomenon proved our hypothesis that subjects performed better on
this test with more trials. Subjects had lower time differences when
classical music was chosen as their second test or third test to
complete, which suggests that listening to classical music results in
higher cognitive flexibility than under the quiet condition. In
addition, we found that listening to music with lyrics as the second or
third test will result in higher value of time differences, which
implies lower cognitive flexibility.

\hypertarget{introduction}{%
\section{Introduction:}\label{introduction}}

Cognitive flexibility is the ability of the brain to shift between doing
different tasks or thinking about different concepts. (Braem and
Egner,2018). In our research, we are investigating whether the auditory
distractions will have an effect on cognitive flexibility. In order to
capture cognitive flexibility, each individual completed three sets of
stroop tests through an app on their electronic devices. During each
test, the respondent was required to identify what colour of a displayed
word was written in while this word itself was a noun for color. We
replicated this process under three different auditory stimulus.

The objective of this statistical analysis is to use appropriate
statistical methods and select a model to test the effects of auditory
distractions on cognitive flexibility. Given in the experiment, there
are two important explanatory variables: distraction levels and the
order of completing those levels. We also used the interaction term
between these two variables to see whether the cognitive flexibility
performance was due the different orders of distraction levels based on
personal choices. Then, we tested all the remaining covariates in the
given dataset to see whether they were statistically significant - for
example, we found that the type of earphones were an important covariate
in this experiment.

We firstly solved some issues that were found in the exploratory data
analysis (EDA), and analyzed the effect of order, distraction level and
the interaction term on the cognitive flexibility performance. We then
tested model assumptions and dealt with outliers. Later, we interpreted
the results and started to test covariates into the model. Lastly, we
included one useful covariate (the type of headphones used) into our
model.

\hypertarget{method}{%
\section{Method:}\label{method}}

The first step we did for this report was to address some issues that we
found in the exploratory data analysis (EDA), such as correcting some
numbers that some subjects had misplaced. Then our analysis emphasized
investigating the research question - ``Does auditory distraction affect
cognitive flexibility''.

\hypertarget{general-settings-in-our-experiment}{%
\subsection{General settings in our
experiment:}\label{general-settings-in-our-experiment}}

The tool we used to measure cognitive flexibility is called EncephalApp
Stroop test. This app has two operational settings - Stroop Off and
Stroop On tests. Subjects are required to successfully pass five runs of
a Stroop Off test before they do a Stroop On test. Here, the Stroop Off
test functions as a practice run, and aims to get subjects more
familiarized with the app and test. Then subjects do a Stroop On test
which contains ten stimuli and a mistake will stop the run.

\hypertarget{reasons-for-choosing-ontime_minus_offtime-as-our-response-variable}{%
\subsection{Reasons for choosing OnTime\_minus\_OffTime as our response
variable:}\label{reasons-for-choosing-ontime_minus_offtime-as-our-response-variable}}

The difference between the Stroop On test and the Stroop Off test will
be the direct and isolated measure for a subject's cognitive
flexibility. (STA490 handout)

\hypertarget{reasons-for-choosing-distraction-levels-as-our-response-variable}{%
\subsection{Reasons for choosing distraction levels as our response
variable:}\label{reasons-for-choosing-distraction-levels-as-our-response-variable}}

Each subject was required to complete three sets of Stroop On and Stroop
Off test under three different levels in order to test whether the
auditory distractions affect people's cognitive flexibility. One level
is ``Quiet'' , where subjects were completing the Stroop test with
earphones but without any background noise or music. Another level was
``Classical Music'' , where subjects were completing the Stroop test
while classical instrumental music - Mozart's Piano Sonata No.8 in A
minor played in the background through the earphones. The last level was
``Music with Lyrics'', where subjects were completing the Stroop test
while a song with a vocal component - Shape of You by Ed Sheeran -
played in the background.

\hypertarget{reasons-for-choosing-order-levels-as-our-response-variable}{%
\subsection{Reasons for choosing order levels as our response
variable:}\label{reasons-for-choosing-order-levels-as-our-response-variable}}

The order of each auditory distraction level was chosen randomly by
participants. This order was included in the results as some subjects
would complete certain subsequent orders faster than the ones before due
to gaining familiarity with the test.

\hypertarget{reasons-for-having-a-random-effect-for-each-individual}{%
\subsection{Reasons for having a random effect for each
individual:}\label{reasons-for-having-a-random-effect-for-each-individual}}

We recorded three observations for each individual, and these three
observations were correlated. Also, each individual had different
cognitive flexibilities to start with. Therefore, we used a random
effect to capture these individual differences and correlations between
observations.

\hypertarget{model_1-with-distraction-level-and-order-as-predictor-variables-and-a-random-effect-for-each-subject}{%
\subsection{Model\_1 with `` Distraction Level'' and `` Order'' as
Predictor Variables And a Random Effect for each
subject:}\label{model_1-with-distraction-level-and-order-as-predictor-variables-and-a-random-effect-for-each-subject}}

In this stage of data analysis, we addressed the research question of
whether or not auditory distraction affected cognitive flexibility.
First, we plotted the measure of cognitive flexibility (which was the
time difference between the Stroop On test and the Stroop Off test)
versus the three different distraction levels to test any potential main
effect that was caused by this explanatory variables. (as Figure 1)

Then later we fitted the most basic model as following:

\hypertarget{y_ijbeta_0beta_1i_classicalbeta_2i_lyricsbeta_3i_orderlevel2beta_4i_orderlevel3-b_ie_ij}{%
\subsubsection{\texorpdfstring{\(Y_{i,j}=\beta_{0}~+~\beta_1I_{Classical}~+~\beta_2I_{Lyrics}+\beta_3I_{OrderLevel=2}+\beta_4I_{OrderLevel=3}+ b_i+e_{ij}\)}{Y\_\{i,j\}=\textbackslash beta\_\{0\}\textasciitilde+\textasciitilde\textbackslash beta\_1I\_\{Classical\}\textasciitilde+\textasciitilde\textbackslash beta\_2I\_\{Lyrics\}+\textbackslash beta\_3I\_\{OrderLevel=2\}+\textbackslash beta\_4I\_\{OrderLevel=3\}+ b\_i+e\_\{ij\}}}\label{y_ijbeta_0beta_1i_classicalbeta_2i_lyricsbeta_3i_orderlevel2beta_4i_orderlevel3-b_ie_ij}}

i stands for each subject(i = 1,2,3,\ldots,72)

j is the number of observations(j = 1,2,3,\ldots,216)

(\beta\_\{0\}) is when subjects choose to have a quiet backgroud
auditory level as their first distraction level.

Y\_\{ij\} is the difference between On time and Off time, which is the
direct measure of cognitive flexibility

bi is the random effect,which is the variation among people

eij is the error term

\hypertarget{model_2-with-distraction-level-and-order-and-the-interaction-term-between-them-as-predictor-variables-and-a-random-effect-for-each-subject}{%
\subsection{Model\_2 with `` Distraction Level'' and `` Order'', and the
interaction term between them as Predictor Variables And a Random Effect
for each
subject:}\label{model_2-with-distraction-level-and-order-and-the-interaction-term-between-them-as-predictor-variables-and-a-random-effect-for-each-subject}}

The most important explanatory variable in our model was the distraction
level which directly answered our research question, followed by the
order of which we conducted these distraction levels. In order to find
any interactions that may have existed between the variables, we tested
the interaction term models with a likelihood ratio test.

\hypertarget{y_ijbeta_0beta_1i_classicalbeta_2i_lyricsbeta_3i_orderlevel2beta_4i_orderlevel3-beta_5i_classicalorderlevel3beta_6i_classicalorderlevel2beta_7i_lyricsorderlevel3beta_8i_lyricsorderlevel2b_ie_ij}{%
\subsubsection{\texorpdfstring{\(Y_{i,j}=\beta_{0}~+~\beta_1I_{Classical}~+~\beta_2I_{Lyrics}+\beta_3I_{OrderLevel=2}+\beta_4I_{OrderLevel=3}+ \beta_5I_{Classical:OrderLevel=3}+\beta_6I_{Classical:OrderLevel=2}+\beta_7I_{lyrics:OrderLevel=3}+\beta_8I_{lyrics:OrderLevel=2}+b_i+e_{ij}\)}{Y\_\{i,j\}=\textbackslash beta\_\{0\}\textasciitilde+\textasciitilde\textbackslash beta\_1I\_\{Classical\}\textasciitilde+\textasciitilde\textbackslash beta\_2I\_\{Lyrics\}+\textbackslash beta\_3I\_\{OrderLevel=2\}+\textbackslash beta\_4I\_\{OrderLevel=3\}+ \textbackslash beta\_5I\_\{Classical:OrderLevel=3\}+\textbackslash beta\_6I\_\{Classical:OrderLevel=2\}+\textbackslash beta\_7I\_\{lyrics:OrderLevel=3\}+\textbackslash beta\_8I\_\{lyrics:OrderLevel=2\}+b\_i+e\_\{ij\}}}\label{y_ijbeta_0beta_1i_classicalbeta_2i_lyricsbeta_3i_orderlevel2beta_4i_orderlevel3-beta_5i_classicalorderlevel3beta_6i_classicalorderlevel2beta_7i_lyricsorderlevel3beta_8i_lyricsorderlevel2b_ie_ij}}

i stands for each subject(i = 1,2,3,\ldots,72)

j is the number of observations(j = 1,2,3,\ldots,216)

(\beta\_\{0\}) is when subjects choose to have a quiet backgroud
auditory level as their first distraction level.

Y\_\{ij\} is the difference between On time and Off time, which is the
direct measure of cognitive flexibility

bi is the random effect,which is the variation among people

eij is the error term

\hypertarget{model_3-is-model_2-with-six-more-potential-covariates}{%
\subsection{Model\_3 is Model\_2 with Six More Potential Covariates
:}\label{model_3-is-model_2-with-six-more-potential-covariates}}

We also collected an additional variable of whether or not the
participating subject was colour-blind. As only one participant stated
that they were colour blind, and the stroop test relied on
differentiating colors, we decided not to use this point of data.

There were six potential covariates in our dataset:

\begin{enumerate}
\def\labelenumi{\arabic{enumi}.}
\tightlist
\item
  Type of headphones used (over-ear or in-ear / noise-cancelling or not
  noise-cancelling)
\item
  Number of years you have studied at an English language institution
  (including 2019-20)
\item
  Whether or not you play video games
\item
  Hours of sleep the night before (for each level of auditory
  distraction)
\item
  Time of day (for each level of auditory distraction)
\item
  Type of device used for the Stroop test app
\end{enumerate}

\hypertarget{model_4with-distraction-level-and-order-and-the-interaction-term-between-them-as-predictor-variables-a-random-effect-for-each-subject-and-a-type-of-earphone-variable}{%
\subsection{Model\_4with `` Distraction Level'' and `` Order'', and the
interaction term between them as Predictor Variables, a Random Effect
for each subject and a type of earphone
variable:}\label{model_4with-distraction-level-and-order-and-the-interaction-term-between-them-as-predictor-variables-a-random-effect-for-each-subject-and-a-type-of-earphone-variable}}

After F-Test on these covariates and some likelihood ratio test,we
included the only useful covariate in to our model. ANd the finalized
model as following:

\hypertarget{y_ijbeta_0beta_1i_classicalbeta_2i_lyricsbeta_3i_orderlevel2beta_4i_orderlevel3-beta_5i_classicalorderlevel3beta_6i_classicalorderlevel2beta_7i_lyricsorderlevel3beta_8i_lyricsorderlevel2beta_9i_in-ear-no-noise-cancelling-beta_10i_over-ear-no-noise-cancelling-beta_11i_over-ear-yes-noise-cancelling-b_ie_ij}{%
\subsubsection{\texorpdfstring{\(Y_{i,j}=\beta_{0}~+~\beta_1I_{Classical}~+~\beta_2I_{Lyrics}+\beta_3I_{OrderLevel=2}+\beta_4I_{OrderLevel=3}+ \beta_5I_{Classical:OrderLevel=3}+\beta_6I_{Classical:OrderLevel=2}+\beta_7I_{lyrics:OrderLevel=3}+\beta_8I_{lyrics:OrderLevel=2}+\beta_9I_{In ear: No Noise Cancelling }+\beta_10I_{Over ear: No Noise Cancelling }+\beta_11I_{Over ear: Yes Noise Cancelling }+b_i+e_{ij}\)}{Y\_\{i,j\}=\textbackslash beta\_\{0\}\textasciitilde+\textasciitilde\textbackslash beta\_1I\_\{Classical\}\textasciitilde+\textasciitilde\textbackslash beta\_2I\_\{Lyrics\}+\textbackslash beta\_3I\_\{OrderLevel=2\}+\textbackslash beta\_4I\_\{OrderLevel=3\}+ \textbackslash beta\_5I\_\{Classical:OrderLevel=3\}+\textbackslash beta\_6I\_\{Classical:OrderLevel=2\}+\textbackslash beta\_7I\_\{lyrics:OrderLevel=3\}+\textbackslash beta\_8I\_\{lyrics:OrderLevel=2\}+\textbackslash beta\_9I\_\{In ear: No Noise Cancelling \}+\textbackslash beta\_10I\_\{Over ear: No Noise Cancelling \}+\textbackslash beta\_11I\_\{Over ear: Yes Noise Cancelling \}+b\_i+e\_\{ij\}}}\label{y_ijbeta_0beta_1i_classicalbeta_2i_lyricsbeta_3i_orderlevel2beta_4i_orderlevel3-beta_5i_classicalorderlevel3beta_6i_classicalorderlevel2beta_7i_lyricsorderlevel3beta_8i_lyricsorderlevel2beta_9i_in-ear-no-noise-cancelling-beta_10i_over-ear-no-noise-cancelling-beta_11i_over-ear-yes-noise-cancelling-b_ie_ij}}

\hypertarget{result}{%
\section{Result:}\label{result}}

\hypertarget{model_1-with-distraction-level-and-order-as-predictor-variables-and-a-random-effect-for-each-subject-1}{%
\subsection{Model\_1 with `` Distraction Level'' and `` Order'' as
Predictor Variables And a Random Effect for each
subject:}\label{model_1-with-distraction-level-and-order-as-predictor-variables-and-a-random-effect-for-each-subject-1}}

According to Figure 1 There is differences between distraction levels,
so we obtained the model statistics as following:

\includegraphics{STA490HW2_files/figure-latex/distraction boxplot-1.pdf}

From figure 1, we observed that the ``quiet'' distraction level has the
widest interquartile range, as well as the highest median. In contrast,
the classical distraction level has the lowest median. The distraction
level ``Lyrics'', had the second lowest median while it had the smallest
maximum values among three distraction levels despite the outliers.
Conclusively, there are some potential effects that auditory levels have
on cognitive flexibility.

some statistc summary for model 1

\begin{verbatim}
##                   numDF denDF   F-value p-value
## (Intercept)           1   140 156.88699  <.0001
## distraction_level     2   140   3.50431  0.0327
## factor(order)         2   140   1.87719  0.1568
\end{verbatim}

Check the model\_1 assumptions:

\includegraphics{STA490HW2_files/figure-latex/unnamed-chunk-1-1.pdf}
\includegraphics{STA490HW2_files/figure-latex/unnamed-chunk-1-2.pdf}

We tested the model assumptions: linearity, the normality of residuals
and homoscedasticity of residuals. Then we found that on the plot of
residuals versus fitted values, the most fitted values were clustered
around 5 to 10, while some outliers were between 10 to 20. We decided to
omit these outliers and refitted the model, resulting in the new model
meeting all the model assumptions better than the old one.

\hypertarget{removing-outliers-and-justification}{%
\subsubsection{Removing outliers and
justification}\label{removing-outliers-and-justification}}

After removing the outliers, we regraphed an boxplot, fitted the model
and checked the model assumptions:
\includegraphics{STA490HW2_files/figure-latex/unnamed-chunk-2-1.pdf}

some statistc summary for model 1 after removing outliers.

\begin{verbatim}
##                   numDF denDF   F-value p-value
## (Intercept)           1   129 233.43014  <.0001
## distraction_level     2   129   2.63326  0.0757
## factor(order)         2   129   1.64903  0.1963
\end{verbatim}

\includegraphics{STA490HW2_files/figure-latex/unnamed-chunk-3-1.pdf}
\includegraphics{STA490HW2_files/figure-latex/unnamed-chunk-3-2.pdf}

We tested the model assumption which was linearity, the normality of
residuals and homoscedasticity of residuals. After omitting 16 outliers,
our model met all the assumptions. Even though we lost some degree of
freedom and had a wider confidence interval after omitting the outliers,
it was necessary to omit them and improve the model's accuracy. For a
model with 216 observations, omitting 16 outliers was deemed acceptable.

\hypertarget{model_2-with-distraction-level-and-order-and-the-interaction-term-between-them-as-predictor-variables-and-a-random-effect-for-each-subject-1}{%
\subsection{Model\_2 with `` Distraction Level'' and `` Order'', and the
interaction term between them as Predictor Variables And a Random Effect
for each
subject:}\label{model_2-with-distraction-level-and-order-and-the-interaction-term-between-them-as-predictor-variables-and-a-random-effect-for-each-subject-1}}

\hypertarget{y_ijbeta_0beta_1i_classicalbeta_2i_lyricsbeta_3i_orderlevel2beta_4i_orderlevel3-beta_5i_classicalorderlevel3beta_6i_classicalorderlevel2beta_7i_lyricsorderlevel3beta_8i_lyricsorderlevel2b_ie_ij-1}{%
\subsubsection{\texorpdfstring{\(Y_{i,j}=\beta_{0}~+~\beta_1I_{Classical}~+~\beta_2I_{Lyrics}+\beta_3I_{OrderLevel=2}+\beta_4I_{OrderLevel=3}+ \beta_5I_{Classical:OrderLevel=3}+\beta_6I_{Classical:OrderLevel=2}+\beta_7I_{lyrics:OrderLevel=3}+\beta_8I_{lyrics:OrderLevel=2}+b_i+e_{ij}\)}{Y\_\{i,j\}=\textbackslash beta\_\{0\}\textasciitilde+\textasciitilde\textbackslash beta\_1I\_\{Classical\}\textasciitilde+\textasciitilde\textbackslash beta\_2I\_\{Lyrics\}+\textbackslash beta\_3I\_\{OrderLevel=2\}+\textbackslash beta\_4I\_\{OrderLevel=3\}+ \textbackslash beta\_5I\_\{Classical:OrderLevel=3\}+\textbackslash beta\_6I\_\{Classical:OrderLevel=2\}+\textbackslash beta\_7I\_\{lyrics:OrderLevel=3\}+\textbackslash beta\_8I\_\{lyrics:OrderLevel=2\}+b\_i+e\_\{ij\}}}\label{y_ijbeta_0beta_1i_classicalbeta_2i_lyricsbeta_3i_orderlevel2beta_4i_orderlevel3-beta_5i_classicalorderlevel3beta_6i_classicalorderlevel2beta_7i_lyricsorderlevel3beta_8i_lyricsorderlevel2b_ie_ij-1}}

The likelihood ratio test for model 1 and model 2

\begin{verbatim}
## Likelihood ratio test
## 
## Model 1: OnTime_minus_OffTime ~ distraction_level + factor(order)
## Model 2: OnTime_minus_OffTime ~ distraction_level * factor(order)
##   #Df  LogLik Df  Chisq Pr(>Chisq)   
## 1   7 -599.52                        
## 2  11 -591.05  4 16.945   0.001981 **
## ---
## Signif. codes:  0 '***' 0.001 '**' 0.01 '*' 0.05 '.' 0.1 ' ' 1
\end{verbatim}

From the likelihood ratio tests, these interaction terms are necessary
and useful since p-value is smaller than 0.05.

\hypertarget{model_3-is-model_2-with-six-more-potential-covariates-1}{%
\subsection{Model\_3 is Model\_2 with Six More Potential Covariates
:}\label{model_3-is-model_2-with-six-more-potential-covariates-1}}

some statistical results for model 3 as well as its model assumptions:

\begin{verbatim}
##                                 numDF denDF   F-value p-value
## (Intercept)                         1   122 246.37102  <.0001
## device                              2    63   3.54159  0.0349
## yrs_english                         1    63   0.05650  0.8129
## video_games                         1    63   1.91946  0.1708
## headphones                          3    63   1.66970  0.1825
## sleep                               1   122   0.07454  0.7853
## start_time                          2   122   0.69288  0.5021
## distraction_level                   2   122   2.68738  0.0721
## factor(order)                       2   122   1.92564  0.1502
## distraction_level:factor(order)     4   122   0.90255  0.4648
\end{verbatim}

\includegraphics{STA490HW2_files/figure-latex/unnamed-chunk-6-1.pdf}

Two Likelihood tests for testing whether headphones and devices are
useful covariates in the model:

\hypertarget{model_4with-distraction-level-and-order-and-the-interaction-term-between-them-as-predictor-variables-a-random-effect-for-each-subject-and-a-type-of-earphone-variable-1}{%
\subsection{Model\_4with `` Distraction Level'' and `` Order'', and the
interaction term between them as Predictor Variables, a Random Effect
for each subject and a type of earphone
variable:}\label{model_4with-distraction-level-and-order-and-the-interaction-term-between-them-as-predictor-variables-a-random-effect-for-each-subject-and-a-type-of-earphone-variable-1}}

\(Y_{i,j}=\beta_{0}~+~\beta_1I_{Classical}~+~\beta_2I_{Lyrics}+\beta_3I_{OrderLevel=2}+\beta_4I_{OrderLevel=3}+ \beta_5I_{Classical:OrderLevel=3}+\beta_6I_{Classical:OrderLevel=2}+\beta_7I_{lyrics:OrderLevel=3}+\beta_8I_{lyrics:OrderLevel=2}+\beta_9I_{In ear: No Noise Cancelling }+\beta_10I_{Over ear: No Noise Cancelling }+\beta_11I_{Over ear: Yes Noise Cancelling }+b_i+e_{ij}\)

\begin{verbatim}
##                                 numDF denDF   F-value p-value
## (Intercept)                         1   125 251.15967  <.0001
## headphones                          3    67   2.91654  0.0405
## distraction_level                   2   125   2.55225  0.0820
## factor(order)                       2   125   1.61650  0.2027
## distraction_level:factor(order)     4   125   0.98177  0.4201
\end{verbatim}

Checking the assumption for this chosen model. We tested the model
assumption which was linearity, the normality, homoscedasticity of
residuals and independence.

From residual versus fitted plot:

\begin{enumerate}
\def\labelenumi{\arabic{enumi}.}
\tightlist
\item
  The residuals ``bounce randomly'' around the 0 line. This suggests
  that the assumption that the linearity is reasonable.
\item
  The residuals roughly form a ``horizontal band'' around the 0 line.
  This suggests that the variances of the error terms are equal.
\item
  No one residual ``stands out'' from the basic random pattern of
  residuals. This suggests that there are no outliers. our model met all
  the assumptions.
\end{enumerate}

From the Normal QQ plot:

since it's roughly a linear straight line, the assumption of nomality is
satisfied.

Also, since the sample of each individual is independent from each
other, so the assumption of independence is satisfied.

\includegraphics{STA490HW2_files/figure-latex/unnamed-chunk-9-1.pdf}
\includegraphics{STA490HW2_files/figure-latex/unnamed-chunk-9-2.pdf}

\begin{verbatim}
##                                                             MLE Std.Error
## (Intercept)                                          5.66374412  1.177385
## headphonesIn-ear headphones; not noise cancelling    1.88060130  1.168413
## headphonesOver-ear headphones; noise cancelling      0.59273973  1.527004
## headphonesOver-ear headphones; not noise cancelling  4.34001199  1.552306
## distraction_levelclassical                           2.08750047  1.974138
## distraction_levellyrics                             -1.14580601  1.588094
## factor(order)2                                      -0.49168172  1.845204
## factor(order)3                                      -1.19980948  1.659760
## distraction_levelclassical:factor(order)2           -4.18730609  2.913373
## distraction_levellyrics:factor(order)2               0.49228651  2.565003
## distraction_levelclassical:factor(order)3           -2.49388609  2.689533
## distraction_levellyrics:factor(order)3              -0.09896269  2.538823
## $\\sigma$                                            2.21944663        NA
## $\\tau$                                              4.16699359        NA
##                                                      DF     t-value
## (Intercept)                                         125  4.81044485
## headphonesIn-ear headphones; not noise cancelling    67  1.60953489
## headphonesOver-ear headphones; noise cancelling      67  0.38817182
## headphonesOver-ear headphones; not noise cancelling  67  2.79584808
## distraction_levelclassical                          125  1.05742357
## distraction_levellyrics                             125 -0.72149753
## factor(order)2                                      125 -0.26646472
## factor(order)3                                      125 -0.72288113
## distraction_levelclassical:factor(order)2           125 -1.43727103
## distraction_levellyrics:factor(order)2              125  0.19192438
## distraction_levelclassical:factor(order)3           125 -0.92725605
## distraction_levellyrics:factor(order)3              125 -0.03897975
## $\\sigma$                                            NA          NA
## $\\tau$                                              NA          NA
##                                                          p-value
## (Intercept)                                         4.253059e-06
## headphonesIn-ear headphones; not noise cancelling   1.122015e-01
## headphonesOver-ear headphones; noise cancelling     6.991195e-01
## headphonesOver-ear headphones; not noise cancelling 6.747627e-03
## distraction_levelclassical                          2.923576e-01
## distraction_levellyrics                             4.719508e-01
## factor(order)2                                      7.903204e-01
## factor(order)3                                      4.711033e-01
## distraction_levelclassical:factor(order)2           1.531395e-01
## distraction_levellyrics:factor(order)2              8.481129e-01
## distraction_levelclassical:factor(order)3           3.555805e-01
## distraction_levellyrics:factor(order)3              9.689687e-01
## $\\sigma$                                                     NA
## $\\tau$                                                       NA
\end{verbatim}

\hypertarget{discussion}{%
\section{Discussion:}\label{discussion}}

\begin{enumerate}
\def\labelenumi{\arabic{enumi}.}
\tightlist
\item
  Though the order was randomly chosen by the subjects, most individuals
  chose the quiet level as their first auditory distraction level,
  leaving the other two distractions levels for second or third. This
  may affect the accuracy of the cognitive flexibility measurements,
  since people tend to get better at the Stroop test with more practice.
  The widest range of this quiet level (as shown in figure 1) implies
  this limitation as well.
\item
  Since the condition of ``Quiet`` is very subjective without any
  official noises tests. The participants are not guaranteed to have
  participated in the test under the exact same noise conditions.
\end{enumerate}

\hypertarget{conclusion}{%
\section{Conclusion:}\label{conclusion}}

From our final model, the average time difference between the on and off
Stroop test for a subject under quiet background noise with in-ear and
noise cancelling earphones was around 6.47 seconds. In general, the
earphones with noise cancelling resulted in lower time differences than
the earphones that did not cancel noises. Moreover, we found that the
average time differences were smaller for order two and order three.
This phenomenon proved our hypothesis that subjects performed better on
this test with more trials. Subjects had lower time differences when
classical music was chosen as their second test or third test to
complete, which suggests that listening to classical music results in
higher cognitive flexibility than under the quiet condition. In
addition, we found that listening to music with lyrics as the second or
third test will result in higher value of time differences, which
implies lower cognitive flexibility.

\hypertarget{appendix}{%
\section{Appendix}\label{appendix}}

\includegraphics{STA490HW2_files/figure-latex/unnamed-chunk-11-1.pdf}
\includegraphics{STA490HW2_files/figure-latex/unnamed-chunk-11-2.pdf}
\includegraphics{STA490HW2_files/figure-latex/unnamed-chunk-11-3.pdf}
\includegraphics{STA490HW2_files/figure-latex/unnamed-chunk-11-4.pdf}

\hypertarget{reference}{%
\section{Reference}\label{reference}}

Braem, S., \& Egner, T. (2018). Getting a Grip on Cognitive Flexibility.
Current Directions in Psychological Science, 27(6), 470--476.
\url{https://doi.org/10.1177/0963721418787475}


\end{document}
